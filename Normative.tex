%% Do not edit unless you really know what you are doing.
\documentclass[twoside,english]{report}
\usepackage[sc]{mathpazo}
\usepackage[scaled=0.9]{helvet}
\renewcommand{\ttdefault}{lmtt}
\usepackage[T1]{fontenc}
\usepackage[latin9]{inputenc}
\usepackage[a4paper]{geometry}
\geometry{verbose,lmargin=2cm,rmargin=2cm}
\usepackage{fancyhdr}
\pagestyle{fancy}
\setcounter{secnumdepth}{3}
\setcounter{tocdepth}{3}
\setlength{\parskip}{\smallskipamount}
\setlength{\parindent}{0pt}
\usepackage{babel}
\usepackage{nomencl}
% the following is useful when we have the old nomencl.sty package
\providecommand{\printnomenclature}{\printglossary}
\providecommand{\makenomenclature}{\makeglossary}
\makenomenclature
\usepackage[unicode=true,
 bookmarks=true,bookmarksnumbered=false,bookmarksopen=false,
 breaklinks=false,pdfborder={0 0 1},backref=false,colorlinks=false]
 {hyperref}
\usepackage{breakurl}

%%COMMENTARE PRIMA DI PUBBLICARE 
%----------------------------------------------------------------
\usepackage{draftwatermark} %per il watermark sullo sfondo
%----------------------------------------------------------------
% Per mettere dei commenti per revisione si possono usare i comandi:
% \unsure per le parti in cui non si � sicuri
% \change per le parti che devono essere cambiate in fase di revisone
% \improvement per le parti che devono essere descritte meglio
% \info per scrivere un commento informativo
% \thiswillnotshow � un commento che viene visto solo nel punto in cui viene piazzato e non nella lista delle note
%----------------------------------------------------------------

\makeatletter


\input{Settings.tex}

\makeatother

\begin{document}
\input{Titlepage.tex}

\input{Titleback.tex}

\pagenumbering{roman}

\begin{multicols}{2}

\printnomenclature{}

\end{multicols}

\tableofcontents{}

\listoffigures


\listoftables

%%COMMENTARE PRIMA DI PUBBLICARE 
%----------------------------------------------------------------
\listoftodos[Notes] %per avere l'elenco delle note
%----------------------------------------------------------------

\clearpage{}

\pagenumbering{arabic}

\setcounter{page}{1}

\global\long\def\diff{\text{d}}


\chapter*{Change Log}
\setlength{\tabcolsep}{0.5em} % for the horizontal padding
{\renewcommand{\arraystretch}{2}% for the vertical padding
	\begin{table}[h]
		\centering
		%\resizebox{\textwidth}{!}
	\end{table}

\info[inline]{Prima di pubblicare, commentare la lista dei todo ed eliminare tutte le note!}
\chapter{Cazzibuffi}
\section{W i Report}

%inserire un'immagine
%\begin{figure}[h]
%	\centering
%	\includegraphics[scale=1]{gfx/nome_dell_immagine}
%	
%	\caption{descrizione sotto la figura}
%	\label{Fig:etichetta_immagine}
%\end{figure}

%inserire un PDF nel pdf
%\begin{figure}[h]
%	\centering
%	\includegraphics[scale=1]{doc/nome_del_documento}
%	
%	\caption{descrizione, se necessaria}
%	\label{Fig:etichetta_documento}
%\end{figure}

\begin{thebibliography}{DOCUMENTI}
	\bibitem{Documento1}
	Citare questo documento con "$\backslash$cite"
	
	\bibitem{Documento 2}
	Citare questo documento come quello sopra
\end{thebibliography}


\end{document}
